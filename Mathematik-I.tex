\documentclass[10pt,a4paper,titlepage,fleqn]{article}
\usepackage[utf8]{inputenc}
\usepackage[german]{babel}
\usepackage[T1]{fontenc}
\usepackage{amsmath}
\usepackage{amsfonts}
\usepackage{amssymb}
\usepackage{tikz}
\usepackage{geometry}
\usepackage{fancyhdr}
\usepackage{titlesec}
\usepackage{enumitem}
\usepackage{wrapfig}
\author{Kirsten Knott} %-> Autor eintragen
\title{Mathematik I}  %-> Titel eintragen
\date{WS 2015/16}		   %-> Datum eintragen

\setcounter{tocdepth}{2}
\linespread{1.4}
\setcounter{page}{0}
\setlist{nolistsep}

%----------------------------------------------------------------------------------------
%	Page Dimensions
%----------------------------------------------------------------------------------------

\geometry{
a4paper,
left=20mm,
right=27mm,
top=34mm,
bottom=26mm,
headsep=6pt
}

%----------------------------------------------------------------------------------------
%	Page Style
%----------------------------------------------------------------------------------------

\pagestyle{fancy}
\fancyhf{}
\rhead{\rightmark}
\lhead{\leftmark}
\fancyfoot[C]{\thepage}
\renewcommand\headrule{
\begin{minipage}{1\textwidth}
\hrule width \hsize \kern 1mm \hrule width \hsize height 2pt 
\end{minipage}}%


%----------------------------------------------------------------------------------------
%	Mengenzeichen
%----------------------------------------------------------------------------------------

\newcommand{\R}{\mathbb{R}}
\newcommand{\Q}{\mathbb{Q}}
\newcommand{\N}{\mathbb{N}}
\newcommand{\Z}{\mathbb{Z}}
\newcommand{\C}{\mathbb{C}}

%----------------------------------------------------------------------------------------
%	TITLE PAGE
%----------------------------------------------------------------------------------------

\newcommand*{\titleGM}{\begingroup % Create the command for including the title page in the document
\hbox{ % Horizontal box
\hspace*{0.2\textwidth}
\rule{1pt}{\textheight}
\hspace*{0.05\textwidth}
\parbox[b]{0.75\textwidth}{

{\noindent\Huge\bfseries Mathematik für \\[0.5\baselineskip] Informatiker I}\\[2\baselineskip] %-> Titel für Deckblatt
{\large \textit{WS 2015/16 Tübingen}}\\[4\baselineskip]	% ->Unterschrift für Deckblatt
{\Large \textsc{Dr. Dorn}}					% -> Mittelzeile Deckblatt

\vspace{0.5\textheight}
{\noindent Kirsten Knott }\\[\baselineskip]  %-> Untere Zeile Deckblatt
}}
\endgroup}


%----------------------------------------------------------------------------------------
%	Sectioning 1.1.a)
%----------------------------------------------------------------------------------------

\renewcommand{\thesubsubsection}{\alph{subsubsection})}

%----------------------------------------------------------------------------------------
%	Sectioning Layouts
%----------------------------------------------------------------------------------------

\titleformat{\section}[block]
{\LARGE\scshape}{\rule{\textwidth}{1pt}\\
\centering §\;\thesection}{7pt}
{}
[\vspace{-10pt}\rule{\textwidth}{0.3pt}\vspace{10pt}]

\titlespacing{\section}{2pt}{.5ex}{.2ex}

\titleformat{\subsubsection}[runin]
{\normalsize\tt}{\thesubsubsection}{4pt}{}

\titlespacing{\subsection}{4pt}{1.5ex}{1.1ex}

\titlespacing{\subsubsection}{1pt}{.5ex}{.8ex}

%----------------------------------------------------------------------------------------

\setlength{\parindent}{0pt}

\begin{document}

\thispagestyle{empty}
\titleGM 
\newpage
\thispagestyle{empty}
\tableofcontents
\newpage

\section{Logik}
	\vspace{0.5em} \textsc{Aussagenlogik} \marginpar{12.10.15}\\
	Eine \underline{\textsf{logische Aussage}} ist ein Satz, der entweder wahr oder falsch (also nie
	 beides zugleich) ist. Wahre Aussagen haben den Wahrheitswert 1 (auch "'wahr"', 
	"'w"', "'true"', "'t"'), falsche den Wert 0 (auch "'falsch"', "'f"', "'false"')\\
	Notation: \quad Aussagenvariablen $A,B,C,\dots \quad A_1,A_2, \dots$\\
	Bsp.: 
	\begin{itemize}
		\setlength\itemsep{-0.1em}
		\item \makebox[8cm][l]{2 ist eine gerade Zahl}(1)
		\item \makebox[8cm][l]{Heute ist Montag}(1)
		\item \makebox[8cm][l]{2 ist eine Primzahl}(1)
		\item \makebox[8cm][l]{12 ist eine Primzahl}(0)
		\item \makebox[8cm][l]{Es gibt unendlich viele Primzahlen}(1)
		\item \makebox[8cm][l]{Es gibt unendlich viele Primzahlzwillinge}(Aussage, 
		 aber unbekannter WHW)
		\item \makebox[8cm][l]{7}(keine Aussage)
		\item \makebox[8cm][l]{Ist 172 eine Primzahl}(keine Aussage)
	\end{itemize}
	Aus einfachen Aussagen \marginpar{14.10.15} kann man durch logische Verknüpfungen 
	(Junktoren, z.B. "'und"', "'oder"') komplizierte bilden.
	Diese werden Ausdrücke genannt. (auch Aussagen sind Ausdrücke)
	Durch sog. \underline{\textsf{Wahrheitstafeln}} gibt man an, wie der Wahrheitswert der
	 zusammengesetzten Aussage durch die Werte der Teilaussagen bedingt ist. \\
	Im Folgenden seien $A,B$ Aussagen\\
	Die wichtigsten Junktoren:
\subsection{Negation}
	Verneinung von $A$: \; \framebox[1.3\width]{$\neg A$}\;
	(auch $\overline{A}\,$), "'nicht $A$"'\\
	\vspace{0.3em}
	ist die Aussage, die genau dann wahr ist, wenn $A$ falsch ist.\\
	\vspace{0.3em}
	Wahrheitstafel:
	$\begin{array}{c|c}
		A & \neg A\\
		\hline
		1 & 0\\
		0 & 1\\
	\end{array}$\\
	\makebox[12cm][l]{Beispiele:\hphantom{$\neg$ }$A$: 6 ist durch 3 teilbar}(1)\\
	\makebox[12cm][l]{\hphantom{Beispiele: }$\neg A$: 6 ist nicht durch 3 teilbar}(0)\\
	\makebox[12cm][l]{\hphantom{Beispiele: }\hphantom{$\neg$}$B$: 4,5 ist eine gerade 
	Zahl}(0)\\
	\makebox[12cm][l]{\hphantom{Beispiele: }$\neg B$: 4,5 ist keine gerade Zahl}(1)
	\newpage
\subsection{Konjunktion}
	Verknüpfung von $A$ und $B$ durch "'und"' \; \framebox[1.3\width]{$A\wedge B$}\\
	ist genau dann wahr, wenn $A$ und $B$ gleichzeitig wahr sind.\\
	$\begin{array}{cc|c}
		A & B & A\wedge B\\
		\hline
		1& 1& 1\\
		1&0&0\\
		0&1&0\\
		0&0&0
	\end{array}$
\subsection{Disjunktion}
	$\begin{array}{lcc|c}
		\text{"'oder"' : }$\framebox[1.3\width]{$A\vee B$}\quad$& A & B & A\vee B\\
		\cline{2-4}
		&1& 1& 1\\
		&1&0&1\\
		&0&1&1\\
		&0&0&0
	\end{array}$
\subsection{XOR}
	$\begin{array}{lcc|c}
		\text{"'entweder oder"' } A \;\mathsf{XOR}\; B, \;$\framebox[1.3\width]{$A\oplus B$}\quad$& A & B & A\oplus B\\
		\cline{2-4}
		\text{(ausschließendes oder)}&1& 1& 0\\
		&1&0&1\\
		&0&1&1\\
		&0&0&0
	\end{array}$
\subsection{Implikation}
	$\begin{array}{lcc|c}
		\text{"'wenn \dots dann"' } $\framebox[1.3\width]{$A\Rightarrow B$}\quad$& A & B & A\Rightarrow B\\
		\cline{2-4}
		\text{(wenn }A\text{ gilt, dann auch }B		&1&1&1\\
		A\text{ impliziert }B						&1&0&0\\
		A\text{ ist hinreichend für }B)				&0&1&1\\
		\text{"'ex falso quodlibet "'}				&0&0&1
	\end{array}$
\subsection{Äquivalenz}
	$\begin{array}{lcc|c}
		\text{"'genau dann wenn"' } $\framebox[1.3\width]{$A
		\Leftrightarrow B$}\quad$& A & B & A\Leftrightarrow B\\
		\cline{2-4}
		\text{(dann und nur dann wenn,}				&1&1&1\\
		\text{äqivalent,}							&1&0&0\\
		\text{if and only if, iff})					&0&1&0\\
													&0&0&1
	\end{array}$\vspace{0.5em}\\
	\textsc{\underline{Festlegung:}} \; \underline{$\neg$} bindet stärker 
	als alle anderen Junktoren ($\neg A \wedge B$ heißt $(\neg A)\wedge B$)
\subsection{Beispiele}
\subsubsection{}
	Wann ist der Ausdruck $(A\vee B)\wedge \neg (A\wedge B)$ wahr?\\
	\quad $\rightarrow$ \; exklusives oder
\subsubsection{}
	$(A\wedge B)\Rightarrow \neg (C\vee A)$\\
	\quad $\rightarrow$ \; $0\;0\;1\;1\;1\;1\;1\;1$
\subsection{Definition "'$\equiv$"'}
	\marginpar{19.10.15} Haben zwei Ausdrücke $\alpha$ und $\beta$ bei jeder 
	Kombination von Wahrheitswerten ihrer Aussagenvariablen den gleichen Wahrheitswert,
	so heißen sie \underline{\textsf{logisch äquivalent}}, man schreibt 
	\framebox[1.3\width]{$\alpha \equiv\beta$} ("'$\equiv$"' ist kein Junktor, 
	entspricht "'$=$"') [ Es gilt: Falls $\alpha \equiv\beta$ gilt, hat der Ausdruck 
	$\alpha \Leftrightarrow\beta$ immer den WHW 1 ]
\subsection{Sätze}
\addtocontents{toc}{\protect\setcounter{tocdepth}{3}}
\subsubsection{Doppelte Negation}
s
\subsubsection{Doppelte Negation}
s
\addtocontents{toc}{\protect\setcounter{tocdepth}{2}}
\end{document}