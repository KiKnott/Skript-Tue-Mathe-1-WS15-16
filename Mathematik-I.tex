\documentclass[10pt,a4paper,titlepage,fleqn]{article}
\usepackage[utf8]{inputenc}
\usepackage[german]{babel}
\usepackage[T1]{fontenc}
\usepackage{amsmath}
\usepackage{amsfonts}
\usepackage{amssymb}
\usepackage{tikz}
\usepackage{geometry}
\usepackage{fancyhdr}
\usepackage{titlesec}
\usepackage{enumitem}
\usepackage{wrapfig}
\usepackage{color}
\author{Kirsten Knott} %-> Autor eintragen
\title{Mathematik I}  %-> Titel eintragen
\date{WS 2015/16}		   %-> Datum eintragen

\setcounter{tocdepth}{2}
\linespread{1.4}
\setcounter{page}{0}
\setlist{nolistsep}

%----------------------------------------------------------------------------------------
%	Page Dimensions
%----------------------------------------------------------------------------------------

\geometry{
a4paper,
left=20mm,
right=27mm,
top=34mm,
bottom=26mm,
headsep=6pt
}

%----------------------------------------------------------------------------------------
%	Page Style
%----------------------------------------------------------------------------------------

\pagestyle{fancy}
\fancyhf{}
\rhead{\rightmark}
\lhead{\leftmark}
\fancyfoot[C]{\thepage}
\renewcommand\headrule{
\begin{minipage}{1\textwidth}
\hrule width \hsize \kern 1mm \hrule width \hsize height 2pt 
\end{minipage}}%


%----------------------------------------------------------------------------------------
%	Eigene Zeichen
%----------------------------------------------------------------------------------------

\newcommand{\R}{\mathbb{R}}
\newcommand{\Q}{\mathbb{Q}}
\newcommand{\N}{\mathbb{N}}
\newcommand{\Z}{\mathbb{Z}}
\newcommand{\C}{\mathbb{C}}
\newcommand{\XOR}{\mathsf{XOR}}
\newcommand*{\TakeFourierOrnament}[1]{{%
\fontencoding{U}\fontfamily{futs}\selectfont\char#1}}
\newcommand*{\danger}{\TakeFourierOrnament{66}}

%----------------------------------------------------------------------------------------
%	TITLE PAGE
%----------------------------------------------------------------------------------------

\newcommand*{\titleGM}{\begingroup % Create the command for including the title page in the document
\hbox{ % Horizontal box
\hspace*{0.2\textwidth}
\rule{1pt}{\textheight}
\hspace*{0.05\textwidth}
\parbox[b]{0.75\textwidth}{

{\noindent\Huge\bfseries Mathematik für \\[0.5\baselineskip] Informatiker I}\\[2\baselineskip] %-> Titel für Deckblatt
{\large \textit{WS 2015/16 Tübingen}}\\[4\baselineskip]	% ->Unterschrift für Deckblatt
{\Large \textsc{Dr. Dorn}}					% -> Mittelzeile Deckblatt

\vspace{0.5\textheight}
{\noindent Kirsten Knott }\\[\baselineskip]  %-> Untere Zeile Deckblatt
}}
\endgroup}


%----------------------------------------------------------------------------------------
%	Sectioning 1.1.a)
%----------------------------------------------------------------------------------------

\renewcommand{\thesubsubsection}{\alph{subsubsection})}

%----------------------------------------------------------------------------------------
%	Sectioning Layouts
%----------------------------------------------------------------------------------------

\titleformat{\section}[block]
{\LARGE\scshape}{\rule{\textwidth}{1pt}\\
\centering §\;\thesection}{7pt}
{}
[\vspace{-10pt}\rule{\textwidth}{0.3pt}\vspace{10pt}]

\titlespacing{\section}{2pt}{.5ex}{.2ex}

\titleformat{\subsubsection}[runin]
{\normalsize\tt}{\thesubsubsection}{4pt}{}

\titlespacing{\subsection}{4pt}{1.5ex}{1.1ex}

\titlespacing{\subsubsection}{1pt}{.5ex}{.8ex}

%----------------------------------------------------------------------------------------

\setlength{\parindent}{0pt}

\begin{document}

\thispagestyle{empty}
\titleGM 
\newpage
\thispagestyle{empty}
\tableofcontents
\newpage

\section{Logik}
	\vspace{0.5em} \textsc{Aussagenlogik} \marginpar{12.10.15}\\
	Eine \underline{\textsf{logische Aussage}} ist ein Satz, der entweder wahr oder falsch (also nie
	 beides zugleich) ist. Wahre Aussagen haben den Wahrheitswert 1 (auch "'wahr"', 
	"'w"', "'true"', "'t"'), falsche den Wert 0 (auch "'falsch"', "'f"', "'false"')\\
	Notation: \quad Aussagenvariablen $A,B,C,\dots \quad A_1,A_2, \dots$\\
	Bsp.: 
	\begin{itemize}
		\setlength\itemsep{-0.1em}
		\item \makebox[8cm][l]{2 ist eine gerade Zahl}(1)
		\item \makebox[8cm][l]{Heute ist Montag}(1)
		\item \makebox[8cm][l]{2 ist eine Primzahl}(1)
		\item \makebox[8cm][l]{12 ist eine Primzahl}(0)
		\item \makebox[8cm][l]{Es gibt unendlich viele Primzahlen}(1)
		\item \makebox[8cm][l]{Es gibt unendlich viele Primzahlzwillinge}(Aussage, 
		 aber unbekannter WHW)
		\item \makebox[8cm][l]{7}(keine Aussage)
		\item \makebox[8cm][l]{Ist 172 eine Primzahl}(keine Aussage)
	\end{itemize}
	Aus einfachen Aussagen \marginpar{14.10.15} kann man durch logische Verknüpfungen 
	(Junktoren, z.B. "'und"', "'oder"') komplizierte bilden.
	Diese werden Ausdrücke genannt. (auch Aussagen sind Ausdrücke)
	Durch sog. \underline{\textsf{Wahrheitstafeln}} gibt man an, wie der Wahrheitswert der
	 zusammengesetzten Aussage durch die Werte der Teilaussagen bedingt ist. \\
	Im Folgenden seien $A,B$ Aussagen\\
	Die wichtigsten Junktoren:
\subsection{Negation}
	Verneinung von $A$: \; \framebox[1.3\width]{$\neg A$}\;
	(auch $\overline{A}\,$), "'nicht $A$"'\\
	\vspace{0.3em}
	ist die Aussage, die genau dann wahr ist, wenn $A$ falsch ist.\\
	\vspace{0.3em}
	Wahrheitstafel:
	$\begin{array}{c|c}
		A & \neg A\\
		\hline
		1 & 0\\
		0 & 1\\
	\end{array}$\\
	\makebox[12cm][l]{Beispiele:\hphantom{$\neg$ }$A$: 6 ist durch 3 teilbar}(1)\\
	\makebox[12cm][l]{\hphantom{Beispiele: }$\neg A$: 6 ist nicht durch 3 teilbar}(0)\\
	\makebox[12cm][l]{\hphantom{Beispiele: }\hphantom{$\neg$}$B$: 4,5 ist eine gerade 
	Zahl}(0)\\
	\makebox[12cm][l]{\hphantom{Beispiele: }$\neg B$: 4,5 ist keine gerade Zahl}(1)
	\newpage
\subsection{Konjunktion}
	Verknüpfung von $A$ und $B$ durch "'und"' \; \framebox[1.3\width]{$A\wedge B$}\\
	ist genau dann wahr, wenn $A$ und $B$ gleichzeitig wahr sind.\\
	$\begin{array}{cc|c}
		A & B & A\wedge B\\
		\hline
		1& 1& 1\\
		1&0&0\\
		0&1&0\\
		0&0&0
	\end{array}$
\subsection{Disjunktion}
	$\begin{array}{lcc|c}
		\text{"'oder"' : }$\framebox[1.3\width]{$A\vee B$}\quad$& A & B & A\vee B\\
		\cline{2-4}
		&1& 1& 1\\
		&1&0&1\\
		&0&1&1\\
		&0&0&0
	\end{array}$
\subsection{XOR}
	$\begin{array}{lcc|c}
		\text{"'entweder oder"' } A \;\mathsf{XOR}\; B, \;$\framebox[1.3\width]{$A\oplus B$}\quad$& A & B & A\oplus B\\
		\cline{2-4}
		\text{(ausschließendes oder)}&1& 1& 0\\
		&1&0&1\\
		&0&1&1\\
		&0&0&0
	\end{array}$
\subsection{Implikation}
	$\begin{array}{lcc|c}
		\text{"'wenn \dots dann"' } $\framebox[1.3\width]{$A\Rightarrow B$}\quad$& A & B & A\Rightarrow B\\
		\cline{2-4}
		\text{(wenn }A\text{ gilt, dann auch }B		&1&1&1\\
		A\text{ impliziert }B						&1&0&0\\
		A\text{ ist hinreichend für }B)				&0&1&1\\
		\text{"'ex falso quodlibet "'}				&0&0&1
	\end{array}$
\subsection{Äquivalenz}
	$\begin{array}{lcc|c}
		\text{"'genau dann wenn"' } $\framebox[1.3\width]{$A
		\Leftrightarrow B$}\quad$& A & B & A\Leftrightarrow B\\
		\cline{2-4}
		\text{(dann und nur dann wenn,}				&1&1&1\\
		\text{äqivalent,}							&1&0&0\\
		\text{if and only if, iff})					&0&1&0\\
													&0&0&1
	\end{array}$\vspace{0.5em}\\
	\textsc{\underline{Festlegung:}} \; \underline{$\neg$} bindet stärker 
	als alle anderen Junktoren ($\neg A \wedge B$ heißt $(\neg A)\wedge B$)
\subsection{Beispiele}
\subsubsection{}
	Wann ist der Ausdruck $(A\vee B)\wedge \neg (A\wedge B)$ wahr?\\
	\quad $\rightarrow$ \; exklusives oder
\subsubsection{}
	$(A\wedge B)\Rightarrow \neg (C\vee A)$\\
	\quad $\rightarrow$ \; $0\;0\;1\;1\;1\;1\;1\;1$
\subsection{Definition "'$\equiv$"'}
	\marginpar{19.10.15} Haben zwei Ausdrücke $\alpha$ und $\beta$ bei jeder 
	Kombination von Wahrheitswerten ihrer Aussagenvariablen den gleichen Wahrheitswert,
	so heißen sie \underline{\textsf{logisch äquivalent}}, man schreibt 
	\framebox[1.3\width]{$\alpha \equiv\beta$} ("'$\equiv$"' ist kein Junktor, 
	entspricht "'$=$"') [ Es gilt: Falls $\alpha \equiv\beta$ gilt, hat der Ausdruck 
	$\alpha \Leftrightarrow\beta$ immer den WHW 1 ]
\subsection{Sätze}
\addtocontents{toc}{\protect\setcounter{tocdepth}{3}}
\subsubsection{Doppelte Negation}
	$A\equiv \neg(\neg A)$
\subsubsection{Kommutativität}
	von $\wedge,\vee,\XOR, \Leftrightarrow$\\
	\hphantom{b) }\danger\; gilt nicht für "'$\Rightarrow$"'
\subsubsection{Assoziativität}
	von $\wedge,\vee,\XOR, \Leftrightarrow$
\subsubsection{Distributivität}
	von $\wedge,\vee$
\subsubsection{Regeln von DeMorgan}\hfill\break
	$\neg (A $\colorbox{orange}{$\wedge$}$ B) \equiv \neg A 
	$\colorbox{orange}{$\vee$}$\neg B\\
	\neg (A $\colorbox{orange}{$\vee$}$ B) \equiv \neg A 
	$\colorbox{orange}{$\wedge$}$\neg B$
\subsubsection{Kontraposition} $A\Rightarrow B \equiv \neg B \Rightarrow \neg A$
\subsubsection{$A\Rightarrow B \equiv \neg A \vee B$ }
\subsubsection{$A\Leftrightarrow B \equiv (A \Rightarrow B)\wedge (B\Rightarrow A)$ }
	\hfill\break
	(Alle Äquivalenzen gelten auch wenn die Aussagenvariablen durch 
	Ausdrücke ersetzt werden.)\\
\addtocontents{toc}{\protect\setcounter{tocdepth}{2}}
	\textsc{\underline{Beweis}}: \;Jeweils mittels Wahrheitstafel,\vspace{0.5em}\\
	$\begin{array}{lc|c|c}
		\text{z.B.: }\texttt{a) } & A & \neg A&\neg(\neg A)\\
		\cline{2-4}
									&0&1&0\\
									&1&0&1
	
	\end{array}$
\subsection{Bemerkung zur Kontraposition}
	(1.9f): $A\Rightarrow B \equiv \underbrace{(\neg B \Rightarrow \neg A)}_{\substack{
	\text{wird \underline{Kontraposition} genannt, wichtig für Beweis.}\\
	\text{ Wird im Sprachgebrauch oft falsch verwendet.}}}$\vspace{0.5em}\\
	\underline{\textsc{Beispiel:}} \; $\underbrace{\text{Pit ist ein Dackel.}}_{A}
	 \;\Rightarrow\; \underbrace{\text{Pit ist ein Hund.}}_{B}$\\
	äquivalent zu: $(\neg B)\Rightarrow (\neg A)$\\
	\hphantom{\underline{\textsc{Beispiel:}} \;} Pit ist kein Hund. $\Rightarrow$
	Pit ist kein Dackel.\\
	aber nicht zu: $B\Rightarrow A$\\
	\hphantom{\underline{\textsc{Beispiel:}} \;} Pit ist ein Hund. $\Rightarrow$
	Pit ist ein Dackel.\\
	\nopagebreak aber nicht zu: $(\neg A)\Rightarrow (\neg B)$\\
	\hphantom{\underline{\textsc{Beispiel:}} \;} Pit ist kein Dackel. $\Rightarrow$
	Pit ist kein Hund.\\
	\underline{\textsc{Beispiel:}} \; Sohn des Logikers / bellende Hunde 
	($\rightarrow$ Folien)
\subsection{Bemerkung (Logisches Umformen)}
	Sei $\alpha$ ein Ausdruck, Ersetzen von Teilausdrücken von $\alpha$ durch logisch
 	äquivalente Ausdrücke liefert einen zu $\alpha$ äquivalenten Ausdruck. So erhält man
 	eventuell kürzere/einfachere Ausdrücke, zum Beispiel:
	\[\neg (A\Rightarrow B) \underset{1.9\,g)}{=}\neg(\neg A\vee B) \underset{1.9\, e)}{=}\neg(\neg A) \wedge (\neg B) \underset{1.9\, a)}{=}A\wedge \neg B\]
\subsection{Definition Tautologie, Kontradiktion, erfüllbar}
	Ein Ausdruck heißt \underline{Tautologie}, wenn er für jede Belegung einer Aussagevariablen immer den
	Wert 1 annimmt.\\
	Hat er immer den Wert 0, heißt er \underline{Kontradiktion}.\\
	Gibt es mindestens eine Belegung der Aussagenvariablen, sodass der Ausdruck den Wert 1 hat, heißt
	er \underline{erfüllbar}.
\subsection{Beispiele}
\subsubsection{} $A\vee \neg A$ Tautologie\\
	\hphantom{a)  } $A\wedge \neg A$ Kontradiktion
\subsubsection{} $\neg (A \Rightarrow B) \Leftrightarrow (A\wedge \neg B)$ Tautologie (vgl. 1.11)\\
	\hphantom{b)  } $(A \Rightarrow B) \Leftrightarrow (\neg A\wedge B)$ Tautologie (vgl. 1.9 g))
\subsubsection{} $A\wedge \neg B$ ist erfüllbar (durch $A=1, \; B=0$\\
	\vspace{0.5em}\underline{\textsc{Prädikatenlogik}}\\
	Eine \underline{Aussageform} ist ein sprachliches Gebilde, das formal wie eine Aussage aussieht, 
	aber nie eine oder mehrere Variablen enthält.\\
	\textsf{Beispiel:} $P(x)\,: \underbrace{x}_{\text{Variable}}\overbrace{<10}^{\text{Prädikat(Eigenschaft)}}$\\
	$Q(x)\,:\;x$ studiert Informatik\\
	\vspace{0.5em}$R(y)\,:\;y$ ist Primzahl und $y^2+2$ ist Primzahl\\
	Eine Aussageform wird zur Aussage, wenn man die Variable durch ein konkretes Objekt ersetzt.\\
	Dies ist nur dann sinnvoll, wenn klar ist, welche Werte für $x$ erlaubt sind, daher wird oft
	die zugelassene Wertemenge mit angegeben. (hier Vorgriff auf Kapitel "'Mengen"').\\
	\textsf{Im Beispiel: } $P(3)$ ist wahr, $P(42)$ falsch.\\
	\hphantom{\textsf{Im Beispiel: } }$R(2)$ ist falsch, $R(3)$ wahr.\\
	Oft ist die Frage interessant, ob es wenigstens ein $x$ gibt, für das $P(x)$ wahr ist, oder ob 
	$P(x)$ sogar für alle zugelassenen $x$ wahr ist.
\subsection{Definition Quantoren}
	Sei $P(x)$ eine Aussageform.
\subsubsection{}
	Die Aussage "'Für alle $x$ (aus einer bestimmten Menge $M$) gilt $P(x)$ ist genau dann wahr, 
	wenn $P(x)$ für alle in Frage \vspace{0.5em} kommenden $x$ wahr ist.\\
	\textsf{Schreibweise: }$\underbrace{\forall}_{\substack{\text{für alle,}\\\text{für jedes}}}
	x \underbrace{\in M}_{\text{aus der Menge }M}\underset{\text{gilt}}{\underset{\downarrow}{:}} 
	\underset{\text{Eigenschaft}}{\underset{\downarrow}{P(x)}}$\\
	auch $\underset{x\in M}{\forall} \; P(X)$ \quad Das Symbol '$\forall$' heißt All-Quantor, die 
	Aussage All-Aussage.
\subsubsection{} Die Aussage "'Es gibt (mindestens) ein $x$ aus der Menge $M$, das die Eigenschaft $P(x)$
	besitzt"' ist wahr genau dann wenn $P(x)$ für mindestens eines der in Frage 
	\vspace{0.5em} kommenden $x$ wahr ist.\\
	\textsf{Schreibweise: } $\underbrace{\exists}_{\substack{\text{es gibt,}\\\text{es existiert}}} x \in M
	\underbrace{:}_{\substack{\text{sodass}\\\text{gilt}}} P(x)$ \quad "'$\exists$"' heißt 
	%TODO Anführungszeichen verbessern
	Existenzquantor, die Aussage Existenzaussage
\subsection{Beispiel/Bemerkung}
	Übungsgruppe G: 
	\begin{tabular}{ccc}
		a & b & c\\
		Anna & Bob & Clara\\
		blond & blond & blond
	\end{tabular}\hfill\break
	$B(x)\,:\;x$ ist blond\\
	$W(x)\,:\;x$ ist weiblich\\
	$B(a)=1,\;\;W(b)=0$
\subsubsection*{1.) } \makebox[12cm][l]{Alle Studenten der Gruppe sind blond}(1)\\
	\hphantom{\textsf{1.)  } }\,\makebox[12cm][l]{$\forall x \in G : \, x\text{ ist blond }
	;\forall x\in G:B(x)$}\vspace{0.5em}(1)\\
	Das bedeutet: 
	\begin{tabular}{ccccc}
	a blond &$\wedge$& b blond &$\wedge$& c blond\\
	${B(a)}_{(1)}$ &$\wedge$& ${B(b)}_{(1)}$ &$\wedge$& ${B(c)}_{(1)}$
	\end{tabular}\\
	$\forall$ ist also eine Verallgemeinerung der Konjunktion.\\
\subsubsection*{2.) } \makebox[12cm][l]{Alle Studenten der Gruppe sind weiblich}(0)\\
	\hphantom{\textsf{2.)  } }\,\makebox[12cm][l]{$\underbrace{W(a)}_1\;\wedge\;
	\underbrace{W(b)}_0\;\wedge\;\underbrace{W(c)}_1$}(0)
\subsubsection*{3.) } \makebox[12cm][l]{Es gibt einen Studenten der Gruppe,
	der weiblich ist.}(1)\\
	\hphantom{\textsf{3.)  } }\,\makebox[12cm][l]{$\exists x \in G: W(x)$}(1)\\
	\hphantom{\textsf{3.)  } }\,\makebox[12cm][l]{$\underbrace{W(a)}_1\;\vee\;
	\underbrace{W(b)}_0\;\vee\;\underbrace{W(c)}_1$}$=1$\\
	$\exists$ ist eine verallgemeinerte Disjunktion.
\subsubsection*{4.) } \makebox[12cm][l]{Aussage $A$: Alle Studenten der Gruppe sind weiblich}(0)\\
	\hphantom{\textsf{4.)  } }\,Verneinung von $A$? \quad $\neg A$\\
	\danger\; \makebox[12cm][l]{Nicht korrekt wäre: Alle Studenten der Gruppe sind männlich.} (0)\\
	\textsf{korrekt: } \makebox[11.35cm][l]{Nicht alle Studenten der Gruppe sind weiblich}(1)\\
	\hphantom{\textsf{korrekt: } }\makebox[11.35cm][l]{Es gibt (mind.) einen Studenten der Gruppe,
	der nicht weiblich ist}(1)\\
	allgemeiner:
\subsection{Negation von All- und Existenzaussagen}
\subsubsection{} $\neg (\forall x \in M: P(x))\equiv \exists x\in M:\neg P(x)$
\subsubsection{} $\neg (\exists x \in M: P(x))\equiv \forall x\in M:\neg P(x)$\\
	(Verallgemeinerung der Regeln von DeMorgan $\rightarrow$ Bsp. 1.15,4)
	\begin{align*}
		\neg (\forall x \in G: W(x))&\equiv \neg(W(a)\wedge W(b) \wedge W(c))\\
		&\underset{DeMorgan}{\equiv}  \neg W(a)\vee \neg W(b) \vee \neg W(c) \\
		&\equiv \exists x \in G : \neg W(x)
	\end{align*}
		%TODO Spacing -> DeMorgan
\end{document}
